
\chapter{Litt om punktlister}\label{kap:punktlister}

Punktlister kan du lage med 
\fbox{\tt $\backslash$itemize}- og
\fbox{\tt $\backslash$enumerate}-kommandoene.

\section{Bruk av \fbox{\tt $\backslash$itemize}}
For å styre avstanden som er mellom punktene i en liste bruker du
\fbox{\tt $\backslash$setlength$\backslash$itemsep\{\}}-kommandoen som
du plasserer inne i mellom 
\fbox{\tt $\backslash$begin\{itemize\}} og
\fbox{\tt $\backslash$end\{itemize\}}.
Følgende kode 

\begin{boxedminipage}{\textwidth}
\begin{verbatim}
\begin{itemize}
  \setlength\itemsep{0mm}
  \item Dette er
  \item en punktliste
  \begin{itemize}
    \setlength\itemsep{-2mm}
    \item som går inn 
    \item flere nivå
  \end{itemize}
  \item før den er tilbake igjen
\end{itemize}
\end{verbatim}
\end{boxedminipage}

gir denne punktlisten: 
\begin{itemize}
  \setlength\itemsep{0mm}
  \item Dette er
  \item en punktliste
  \begin{itemize}
     \setlength\itemsep{-2mm}
     \item som går inn 
     \item flere nivå
   \end{itemize}
 \item før den er tilbake igjen
\end{itemize}

Du kan også velge helt selv hva du vil at punktnummereringen skal
være. For eksempel denne selvvalgte nummereringen

\begin{boxedminipage}{\textwidth}
\begin{verbatim}
\begin{itemize}
\item [3)] På tredje plass ...
\item [2)] På andre plass ....
\item [1)] På første plass ...
\end{itemize}
\end{verbatim}
\end{boxedminipage}

gir denne listen

\begin{itemize}
\item [3)] På tredje plass ...
\item [2)] På andre plass ....
\item [1)] På første plass ...
\end{itemize}

\newpage

\section{Bruk av \fbox{\tt $\backslash$enumerate}}
Listen {\tt enumerate} kan brukes på mange måter.
\subsection{Tallbasert liste, standardversjonen}
Ved å bruke {\tt enumerate} på følgende måte

\begin{boxedminipage}{\textwidth}
\begin{verbatim}
\begin{enumerate}
\item Dette er en standard
\item tallbasert liste
\end{enumerate}
\end{verbatim}
\end{boxedminipage}

får vi  denne listen
\begin{enumerate}
\item Dette er en standard
\item tallbasert liste
\end{enumerate}

\subsection{Bostavbasert liste}
Enumerate-listen kan endre ved å spesifisere 
andre tellemåter i hakeparentes bak {\tt
  $\backslash$begin\{$\backslash$enumerate}\}. Følgende kode

\begin{boxedminipage}{\textwidth}
\begin{verbatim}
\begin{enumerate}[a)]
\item Dette er en 
\item bokstavbasert liste
\end{enumerate}
\end{verbatim}
\end{boxedminipage}

gir denne listen
\begin{enumerate}[a)]
\item Dette er en 
\item bokstavbasert liste
\end{enumerate}

\subsection{Romertallbasert liste}
Alternativt kan vi bruke romertall som vist under.
Følgende kode

\begin{boxedminipage}{\textwidth}
\begin{verbatim}
\begin{enumerate}[i.]
\item Dette er en 
\item romertall liste
\item 
\item 
\item 
\end{enumerate}
\end{verbatim}
\end{boxedminipage}

gir denne liste
\begin{enumerate}[i.]
\item Dette er en 
\item romertall liste
\item 
\item 
\item 
\end{enumerate}


