
\chapter{Litt om tabeller}\label{kap:tabeller}

\section{Bruk av \fbox{table}- og \fbox{tabular}-miljøene}

Tabeller kan være vanskelig i \LaTeX. Under er det vist noen eksempler
på tabeller slik at du kan bruke de som et utgangspunkt for dine egne
tabeller.
Husk at
tabelltekst skal stå over tabellen, imotsetning til figurtekst som
står under.

Følgende kode gir tabell~\ref{tab:enkel_tab}
som resultat. 

\begin{boxedminipage}{\textwidth}
\begin{verbatim}
\begin{table}[H]
  \centering
  \caption{Enkel tabell.}
  \begin{tabular}{|l|r c|r|c|}\hline
    Venstrejustert & Høyrejustert & Sentrert 
    &  Høyrejustert & Sentrert \\\hline\hline 
    1.3434 & 0.55677 & 0.2 & 1.45   & 0.3     \\
    2.5667 & 0.66    & 0.4 & 100.01 & 1.43435 \\\hline
  \end{tabular}
  \label{tab:enkel_tab}
\end{table}
\end{verbatim}
\end{boxedminipage}

\begin{table}[H]
  \centering
  \caption{Enkel tabell.}
  \begin{tabular}{|l|r c|r|c|}\hline
    Venstrejustert & Høyrejustert & Sentrert 
    &  Høyrejustert & Sentrert \\\hline\hline 
    1.3434 & 0.55677 & 0.2 & 1.45     & 0.3  \\
    2.5667 & 0.66       & 0.4 & 100.01 & 1.43435  \\\hline
  \end{tabular}
  \label{tab:enkel_tab}
\end{table}
Hvert element i tabellen skilles med et \&-tegn, og det kan være
nyttig å organiserer disse \&-tegnene oppå hverandre slik at tabellen
er lettere å lese i .tex-fila.  


Detaljene som er vist i følgende linje i koden over

\begin{boxedminipage}{\textwidth}
\begin{verbatim}
  \begin{tabular}{|l|r c|r|c|}\hline
\end{verbatim}
\end{boxedminipage}

betyr at tabellen skal inneholde 5 kolonner hvor
\begin{itemize}
\item  symbolet \fbox{|} betyr loddrett tabellstrek, 
\item \fbox{\tt l} betyr {\em left} og dermed venstrejustert
\item \fbox{\tt r} betyr {\em right} og dermed høyrejustert
\item \fbox{\tt c} betyr {\em center} og dermed sentrert
\item kommanoden \fbox{\tt $\backslash$hline} betyr {\em horisontal
    line} og dermed vannrett tabellstrek
\end{itemize}

\newpage

Koden for en litt mer avansert tabell er vist under og i 
tabell~\ref{tab:avansert_tab}. Denne inneholder både 
{\tt $\backslash$multirow}, 
{\tt $\backslash$multicolumn}, 
{\tt $\backslash$cellcolor}, 
 og{\tt $\backslash$cline}. 

\begin{boxedminipage}{160mm}
\begin{verbatim}
\begin{table}[H]
  \centering
  \renewcommand{\arraystretch}{1.2}
  \caption{Parametre brukt i simuleringen.}
  \begin{tabular}{|c||c|c||c|c|c|}\hline
    \multirow{2}{*}{Controller} & \multicolumn{2}{c||}{Variables} 
                                & \multicolumn{3}{c|}{Parameters}\\
    \cline{2-6}
    no. & \cellcolor[gray]{0.8} & & & &\\[-4mm]
          & \cellcolor[gray]{0.8}  $h_{high}$ &$u_{high}$  & $k_{s}^{u}$  
                                & $V_{max}^{u}$ &    $K_{M}^{u}$  \\[1mm]\hline
    1 & \cellcolor[gray]{0.8} 0.55 & 0.49 & 0.01 & 0.0071  & 0.137 \\
    2 & \cellcolor[gray]{0.8} 0.66 & 0.47 & 0.01 & 0.0177  & 0.797 \\\hline
  \end{tabular}
  \label{tab:avansert_tab}
\end{table}
\end{verbatim}
\end{boxedminipage}

\begin{table}[H]
  \centering
  \renewcommand{\arraystretch}{1.2}
  \caption{Parametre brukt i simuleringen.}
  \begin{tabular}{|c||c|c||c|c|c|}\hline
    \multirow{2}{*}{Controller} & \multicolumn{2}{c||}{Variables} & \multicolumn{3}{c|}{Parameters}\\
    \cline{2-6}
    no. & \cellcolor[gray]{0.8} & & & &\\[-4mm]
          & \cellcolor[gray]{0.8}  $h_{high}$ &$u_{high}$  & $k_{s}^{u}$  
                                & $V_{max}^{u}$ &    $K_{M}^{u}$  \\[1mm]\hline
    1 & \cellcolor[gray]{0.8} 0.55 & 0.49 & 0.01 & 0.0071  & 0.137 \\
    2 & \cellcolor[gray]{0.8} 0.66 & 0.47 & 0.01 & 0.0177  & 0.797 \\\hline
  \end{tabular}
  \label{tab:avansert_tab}
\end{table}

Legg merke til at kommadoen
{\tt $\backslash$renewcommand\{$\backslash$arraystretch\}\{1.2\}} 
øker høyden med 20\% for hver rekke i tabellen. Legg også merke til at
du kan flytte neste linje opp eller ned med å legge skrive en høyde i
hakeparentes, f.eks.
{\tt  [-4mm]}, bak linjeskiftkommandoen $\backslash$$\backslash$.