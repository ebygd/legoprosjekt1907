

\chapter{Litt om ligninger}\label{kap:ligninger}

\section{Bruk av \fbox{\tt equation}-miljøet}\label{delkap:equation}
For å skrive ligninger i {\LaTeX} brukes \fbox{\tt equation}-miljøet
på følgende måte: 

\begin{boxedminipage}{\textwidth}
\begin{verbatim}
\begin{equation}
  \label{eq:sinus}
  y(t)=\sin(\omega t)
\end{equation}
\end{verbatim}
\end{boxedminipage}

som resulterer i følgende ligning hvor ligningsnummeret~\eqref{eq:sinus}
kommer automatisk opp på høyre side av ligningen.
\begin{equation}
  \label{eq:sinus}
  y(t)=\sin(\omega t)
\end{equation}


\section{Bruk av \fbox{\tt align}-miljøet}\label{delkap:align}
For å sette flere ligninger over hverandre er 
\fbox{\tt align} nyttig. Denne benytter tegnet {\&} for å markere
hvilken plass i ligningene som skal {\em alignes}.
I  eksemplet under er det likhetstegnet og 
{\tt $\backslash$downarrow}  som skal stå
over hverandre. For hver linjeslutt {\color{red}(utenom den siste
  linjen)} må du indikere det med
$\backslash\backslash$.

\begin{boxedminipage}{\textwidth}
\begin{verbatim}
\begin{align}
  y(t) & = \sin(\omega t) \tag{\ref{eq:sinus}} \\
       & \downarrow \notag \\
  y(t) & = \cos \Bigl( \omega t - \frac{\pi}{2} \Bigr) 
         \label{eq:cosinus}
\end{align}
\end{verbatim}
\end{boxedminipage}
 
Resultatet av koden er vist under
\begin{align}
  y(t) & = \sin(\omega t) \tag{\ref{eq:sinus}}\\
        & \downarrow \notag\\
  y(t) & = \cos \Bigl(\omega t - \frac{\pi}{2}\Bigr) \label{eq:cosinus}
\end{align}
Noen kommentarer til resultatet:
\begin{itemize}
\item Legg merke til at ligningsnummeret~\eqref{eq:sinus} til den øverste ligningen er det
  samme i ligningen i forrige delkapittel, og dette er gjort
  ved å bruke \fbox{\tt $\backslash$tag\{\}}-funksjonen som manuelt {\em
    tag}'er en ligning med ligningsnummeret til en annen ligning ved
  bruk av \fbox{\tt $\backslash$ref\{\}}-funksjonen fra
  kapittel~\ref{delkap:ref}. Denne måten å lage ligningsnummer på
  brukes bare når ligningene er identiske, slik som de er i dette
  tilfelle. 

\item Videre ser du at {\tt $\backslash$downarrow}-pilen 
  ikke har ligningsnummer (som er jo logisk), og
  dette løses med \fbox{\tt $\backslash$notag}-funksjonen.

\item Legg også merke til at selve sinus- og cosinusfunksjonene i ligningene~\eqref{eq:sinus} 
  og \eqref{eq:cosinus} 
  ser ut som vanlig stående tekst siden de skrives med kommandoen 
  \fbox{\tt $\backslash$sin} og \fbox{\tt $\backslash$cos}.

\item Legg merke til at siste linje IKKE skal ha
  $\backslash\backslash$ på siste linje. Dersom du glemmer dette, vil det dukke opp
  en tom linje med ligningsnummer.
\end{itemize}



\section{Bruk av matematikkmodus i rapporttekst, \$x\$}
Når du i teksten skriver om variabler som brukes i ligningene må du
huske å skrive disse i matematikkmodus, det vil si med dollartegn
foran og etter symbolet. I ligning~\eqref{eq:sinus} brukes for eksempel tiden
$t$, hvor {\LaTeX}-koden for å skrive tidsvariabel i matematikkmodus er \fbox{\tt \$t\$}.

Unngå å skrive ligninger i selve teksten. Dette fordi det ofte er
vanskelig å lese, og du kan heller ikke referere til ligningen senere
i teksten. Bruk derfor {\tt $\backslash$begin\{equation\}} og 
  {\tt $\backslash$end\{equation\}} for alle ligninger.


\section{Bruk av vanlig tekst i ligninger, \fbox{\tt $\backslash$mathrm\{ \}}}
Det er ikke ofte du har bruk for å skrive vanlig tekst i ligningene,
men skal du gjøre dette kan du bruke 
\fbox{\tt $\backslash$mathit\{\}} eller 
\fbox{\tt $\backslash$mathrm\{\}} kommadoene. 
Et eksempel på dette er vist i ligning~\eqref{eq:sin_og_cos}, hvor 
\fbox{\tt $\backslash$underbrace\{\}}-kommandoen og 
\fbox{\tt $\backslash$hspace*\{\}}-kommandoen  er brukt i \fbox{\tt
  align}-miljøet. 

\begin{boxedminipage}{160mm}
\begin{verbatim}
\begin{align}
  y(t) & = \underbrace{\sin(\omega t)} 
          + \underbrace{\cos(2 \pi f t)} \label{eq:sin_og_cos}\\
       & \hspace*{5mm}\mathrm{Roman font} 
         \hspace*{7mm}\mathit{Italic~font}\notag
\end{align}
\end{verbatim}
\end{boxedminipage}

som gir følgende resultat (legg merke til at du må bruke tilde for å
lage mellomrom)
\begin{align}
  y(t)  &= \underbrace{\sin(\omega t)} 
          + \underbrace{\cos(2 \pi f t)}   \label{eq:sin_og_cos} \\
          & \hspace*{5mm}\mathrm{Roman font}
            \hspace*{7mm}\mathit{Italic~font} \notag
\end{align}

\section{Bruk av \fbox{\tt $\backslash$hspace*\{ \}}}\label{delkap:hspace}
I eksempelet over ble kommandoen \fbox{\tt $\backslash$hspace*\{\}}
brukt.  Denne betyr {\em horisontal space}, og
tilsvarende finnes det også \fbox{\tt $\backslash$vspace*\{\}} betyr 
{\em vertical  space}. For at ikke {\LaTeX} skal ignorere dine ønsker
å horisontal og vertikal avstand, MÅ du bruke {\tt *} slik som vist i
koden over. 

Husk at du kan flytte både til høyre/venstre og ned/opp ved å
bruke hhv. positive og negative verdier, og du kan angi forflytningen
i mm, cm, eller punkter pt.

