
\chapter*{Sammendrag}  
\addcontentsline{toc}{chapter}{\protect\numberline{}Sammendrag} 
Det er i hovedsak 4 hensikter med dette dokumentet og de tilhørende filene
i mappen \fbox{Rapport\_LaTeX}:

\begin{enumerate}
\item Det fungerer som et
skall for å skrive ING100-rapporten  i {\LaTeX}. 
Delkapitlene i kapittel~\ref{kap:prosjekt01} til
\ref{kap:prosjekt05} er bare et forslag til inndeling.

\item Det gir et  eksempel på hvordan et prosjekt
\todo[size=\footnotesize]{Underveis i kapittelet brukes slike oransje
  kommentarbokser til å fremheve noen poenger om rapportskriving.}
kan dokumenteres i kapittel~\ref{kap:turtall}.
Grunnen til at dette prosjektet med resultater serveres deg i sin
helhet er at du har stor nytte av å anvende turtallsregulering i
prosjektene du gjennomfører. Stoffet kan oppleves som vanskelig, og 
derfor fremstår dokumentasjonen av prosjektet som relativt
omfattende. Poenget er at du skal dokumentere på lignende måte dine
egne prosjekt i ING100. Tenk derfor etter mens du leser gjennom
kapittel~\ref{kap:turtall} om noe er uklart, eller om du opplever det
som overforklart. ING100-rapporten din skal nemlig
skrives for en person som har samme bakgrunn som deg selv, og
det er viktig at du tenker gjennom hvordan du formulerer deg og på
hvilket nivå du legger forklaringene. 

\item Det fungerer som et oppslagshefte for tips til
{\LaTeX}. Gjelder kapittel~\ref{kap:referering} til \ref{kap:kode}.

\item Det gir et krasjkurs i Overleaf. Gjelder kapittel~\ref{kap:overleaf}.


\end{enumerate}
